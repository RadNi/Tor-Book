\chapterimage{chapter_head_2.pdf} % Chapter heading image

\chapter{Preliminaries}
\label{sec:torspec-0}
\section{Ciphers}
%\index{Paragraphs of Text}
\subsection{Stream Cipher}
For stream ciphers Tor uses 128-bit AES in counter mode, with an IV of all 0 bytes.
\\
In section \ref{sec:AES} we provide some notes about AES counter mode. Tor usage for AES counter mode with EVP APIs is located \href{https://gitweb.torproject.org/tor.git/tree/src/lib/crypt_ops/aes_openssl.c}{here} .
\\
For a public-key cipher, uses RSA with 1024-bit keys and a fixed exponent of 65537 and OAEP-MGF1 padding format, with SHA-1 as its digest function. Further information about RSA explained in section \ref{sec:RSA}.

%TODO @radni add ECC and Diffie Hellman and ... documentation here
%src/lib/crypt_ops/crypto_curve25519.c +190 curve25519 key generation
%   We also use the Curve25519 group and the Ed25519 signature format in
%   several places.
%
%   For Diffie-Hellman, unless otherwise specified, we use a generator
%   (g) of 2.  For the modulus (p), we use the 1024-bit safe prime from
%   rfc2409 section 6.2 whose hex representation is:
%
%     "FFFFFFFFFFFFFFFFC90FDAA22168C234C4C6628B80DC1CD129024E08"
%     "8A67CC74020BBEA63B139B22514A08798E3404DDEF9519B3CD3A431B"
%     "302B0A6DF25F14374FE1356D6D51C245E485B576625E7EC6F44C42E9"
%     "A637ED6B0BFF5CB6F406B7EDEE386BFB5A899FA5AE9F24117C4B1FE6"
%     "49286651ECE65381FFFFFFFFFFFFFFFF"
%
%   As an optimization, implementations SHOULD choose DH private keys (x) of
%   320 bits.  Implementations that do this MUST never use any DH key more
%   than once.
%   [May other implementations reuse their DH keys?? -RD]
%   [Probably not. Conceivably, you could get away with changing DH keys once
%   per second, but there are too many oddball attacks for me to be
%   comfortable that this is safe. -NM]
%
%   For a hash function, unless otherwise specified, we use SHA-1.
%
%   KEY_LEN=16.
%   DH_LEN=128; DH_SEC_LEN=40.
%   PK_ENC_LEN=128; PK_PAD_LEN=42.
%   HASH_LEN=20.
%
%   We also use SHA256 and SHA3-256 in some places.
%
%   When we refer to "the hash of a public key", unless otherwise
%   specified, we mean the SHA-1 hash of the DER encoding of an ASN.1 RSA
%   public key (as specified in PKCS.1).
%
%   All "random" values MUST be generated with a cryptographically
%   strong pseudorandom number generator seeded from a strong entropy
%   source, unless otherwise noted.

%TODO @radni src/lib/crypto_ops/crypto_rand.c



%------------------------------------------------

%\section{Citation}\index{Citation}
%
%This statement requires citation \cite{article_key}; this one is more specific \cite[162]{book_key}.
%
%%------------------------------------------------
%
%\section{Lists}\index{Lists}
%
%Lists are useful to present information in a concise and/or ordered way\footnote{Footnote example...}.
%
%\subsection{Numbered List}\index{Lists!Numbered List}
%
%\begin{enumerate}
%\item The first item
%\item The second item
%\item The third item
%\end{enumerate}
%
%\subsection{Bullet Points}\index{Lists!Bullet Points}
%
%\begin{itemize}
%\item The first item
%\item The second item
%\item The third item
%\end{itemize}
%
%\subsection{Descriptions and Definitions}\index{Lists!Descriptions and Definitions}
%
%\begin{description}
%\item[Name] Description
%\item[Word] Definition
%\item[Comment] Elaboration
%\end{description}
