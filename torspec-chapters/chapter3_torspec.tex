\chapter{Cell Packet Format}

\textbf{Cells}\\
\\
Cell is the basic unit for communication in onion routers and onion proxies. Cells have fixed-width size of \textbf{512 bytes}, all of them are consisted by \textit{header} and \textit{payload}.
\section{Cell versions}

Different versions of tor connection have different cell structure:
\\
\begin{table}[h!]
  \begin{center}
    \caption{Version 1 cell packet format}
    \label{tab:CPFV1}
    \begin{tabular}{c|c|l} % <-- Alignments: 1st column left, 2nd middle and 3rd right, with vertical lines in between
      \textbf{Command} & \textbf{Length} & \textbf{Description}\\
	\toprule
      CircID & CIRCID\_LEN bytes & \begin{tabular}{l}
      Determines which circuit (if any)\\
      the connection is associated with\\
      \end{tabular}\\
      \midrule
      Command & 1 byte & \begin{tabular}{l}
      Command field tells what\\
      to do with the cell\\
      \end{tabular}\\      
      \midrule
      Payload (padded with padding bytes) & PAYLOAD\_LEN bytes & \begin{tabular}{l}interpretation is based on the command field\\ \end{tabular}
   		
    \end{tabular}
  \end{center}
\end{table}
\\
\textbf{Version 2 and higher}
\\
All cells have the same format as \textit{Version 1} except the variable length connections.
\begin{table}[h!]
  \begin{center}
    \caption{Version 2 and higher cell format}
    \label{tab:CPFV2U}
    \begin{tabular}{c|c|l} 
      \textbf{Command} & \textbf{Length} & \textbf{Description}\\
	\toprule
      CircID & CIRCID\_LEN bytes & \begin{tabular}{l}
      Determines which circuit (if any)\\
      the connection is associated with\\
      \end{tabular}\\
      \midrule
      Command & 1 byte & \begin{tabular}{l}
      Command field tells what\\
      to do with the cell\\
      \end{tabular}\\     
      \midrule
      Payload (padded with padding bytes) & PAYLOAD\_LEN bytes & \begin{tabular}{l}interpretation on the command field\\ \end{tabular}\\
     \midrule
     Length & 2 bytes big-endian integer & \begin{tabular}{l}			Determines the length of cell\\
     it will be specified for processing\\
      with a command byte ( explained blow )\\ \end{tabular}\\
      \midrule
      Payload (some commands MAY pad) & \textit{Length} bytes & \begin{tabular}{l}interpretation is based on the command\\
field, some commands MAY be padded\\      
       \end{tabular}\\
    \end{tabular}
  \end{center}
\end{table}
\\
\\
\subsection{Headers and Commands}\hfill\\
\textbf{CircID Header}
\\
 The header includes a circuit identifier (\textit{circID}) that
specifies which circuit the cell refers to (many circuits can
be multiplexed over the single TLS connection), and a command to describe what to do with the cell's payload. (Circuit
identifiers are connection-specific: each circuit has a different \textit{circID} on each OP/OR or OR/OR connection it traverses.)
\\
\\
Based on their command, cells are either \textbf{control cells}, which
are always interpreted by the node that receives them, or \textbf{relay cells}, which carry end-to-end stream data.
\\
\\
The 'Command' field of a fixed-length and varibale-length cell holds one of the following
values:
\begin{table}[h!]
\caption{Type of commands}
    \label{tab:commands}
    \begin{tabular}{c|c|l} 
      \textbf{Value} & \textbf{Command (fixed-length)} & \textbf{Description and References}\\
	\toprule
0 & PADDING & padding cell (Sec 7.2)\\
1 & CREATE & Create a circuit (Sec 5.1)\\
2 & CREATED & Acknowledge create (Sec 5.1)\\
3 & RELAY & End-to-end data (Sec 5.5 and Sec 6)\\
4 & DESTROY & destroy a circuit (Sec 5.4)\\
5 & CREATE\_FAST & Create a circuit, no PK (Sec 5.1)\\ 
6 & CREATED\_FAST & Circuit created, no PK (Sec 5.1)\\
8 & NETINFO & Time and address info (Sec 4.5)\\
9 & RELAY\_EARLY & End-to-end data (Sec 5.6)\\
10 & CREATE2 & Extended CREATE cell (Sec 5.1)\\
11 & CREATED2 & Extended CREATED cell (Sec 5.1)\\
12 & PADDING\_NEGOTIATE & Padding negotiation (Sec 7.2)\\
\toprule
\multicolumn{3}{c}{\textbf{Command (variable-length)}}\\
\toprule
7 & VERSIONS & Negotiate proto version (Sec 4)\\
128 & VPADDING & Variable-length padding (Sec 7.2)\\
129 & CERTS & Certificates (Sec 4.2)\\
130 & AUTH\_CHALLENGE & Challenge value (Sec 4.3)\\
131 & AUTHENTICATE & Client authentication (Sec 4.5)\\
132 & AUTHORIZE & Client authorization (Reserved for future use)\\
\end{tabular}
\end{table}
\\
\\
\textbf{Control cells}
\\
\\
The control cell commands are: \textit{padding} (currently used for keepalive,
but also usable for link padding); \textit{create} or \textit{created} (used to
set up a new circuit); and \textit{destroy} (to tear down a circuit).
\newline
\begin{center}
\begin{tabular}{cccccccc}
2 & 1 & \multicolumn{6}{c}{509 bytes}\\
\hline
\vline CircId & \vline CMD \vline & \multicolumn{6}{c}{DATA}\vline\\
\hline
\end{tabular}
\end{center}
\textbf{Relay cells}
\\
\\
Relay cells have an additional header (the relay header) at
the front of the payload, containing a \textit{streamID} (stream identifier: many streams can be multiplexed over a circuit); an
end-to-end \textit{checksum} for integrity checking; the \textit{length} of the
relay payload; and a relay \textit{command}. The entire contents of
the relay header and the relay cell payload are encrypted or
decrypted together as the relay cell moves along the circuit,
using the \textbf{128-bit AES} cipher in counter mode to generate a
cipher stream.
\newline
\\
\textbf{Relay commands}
\\
\\
The relay commands are: relay \textit{data} (for data flowing down the stream), relay \textit{begin} (to open a stream), relay \textit{end} (to close a stream cleanly), relay \textit{teardown} (to close a broken stream), relay \textit{connected} (to notify the OP that a relay
begin has succeeded), relay \textit{extend} and relay \textit{extended} (to extend the circuit by a hop, and to acknowledge), relay \textit{truncate}
and relay \textit{truncated} (to tear down only part of the circuit, and
to acknowledge), relay \textit{sendme} (used for congestion control),
and relay \textit{drop} (used to implement long-range dummies).
\newline
\\
\newline
\begin{center}
\begin{tabular}{ccccccccccccc}
2 & 1 & 2 & 6 & 2 & 1 & \multicolumn{6}{c}{498 bytes}\\
\hline
\vline CircId & \vline Relay \vline & StreamID \vline & Digest \vline & Length \vline & CMD \vline & \multicolumn{6}{c}{DATA}\vline\\
\hline
\end{tabular}
\end{center}
\subsection{Further explanations}\hfill\\
\textbf{Version cells}
\\
\\
Most variable-length cells MAY be padded with padding bytes, except
for \textit{VERSIONS} cells, which MUST NOT contain any additional bytes.
(The payload of \textit{VPADDING} cells consists of padding bytes.)
\textit{version} cell is used in negotiation phase.
\\
On a \textbf{version 2} connection, variable-length cells are indicated by a
command byte equal to 7 ("\textit{VERSIONS}"). On a \textbf{version 3 or
higher} connection, variable-length cells are indicated by a command
byte equal to 7 ("\textit{VERSIONS}"), or greater than or equal to 128.
\\
\\
\textbf{CircID}
\\
\\
CIRCID\_LEN is 2 for link protocol versions 1, 2, and 3. CIRCID\_LEN
is 4 for link protocol version 4 or higher. The first VERSIONS cell,
and any cells sent before the first VERSIONS cell, always have
CIRCID\_LEN == 2 for backward compatibility.
The CircID field determines which circuit, if any, the cell is
associated with.
\\
\\
\textbf{Interpretation of payload}
\\
\\
The interpretation of 'Payload' depends on the type of the cell.\\\\
\textit{VPADDING/PADDING}: Payload contains padding bytes.\\
\textit{CREATE/CREATE2}: Payload contains the handshake challenge.\\
\textit{CREATED/CREATED2}: Payload contains the handshake response.\\
\textit{RELAY/RELAY\_EARLY}: Payload contains the relay header and relay body.\\
\textit{DESTROY}: Payload contains a reason for closing the circuit.
(see 5.4)\\
Upon receiving any other value for the command field, an OR must
drop the cell. Since more cell types may be added in the future, ORs
should generally not warn when encountering unrecognized commands.
\\
\\
\textbf{Padding and padding bytes}
\\
\\
The cell is padded up to the cell length with padding bytes and receivers MUST ignore padding bytes.\\
Senders set padding bytes depending on the cell's command:\\\\
\textit{VERSIONS}: Payload MUST NOT contain padding bytes.\\
\textit{AUTHORIZE}: Payload is unspecified and reserved for future use.\\
\\
Other \textbf{variable-length} cells:\\\\
Payload MAY contain padding bytes at the end of the cell.
Padding bytes SHOULD be set to NULL.\\
\textit{RELAY/RELAY\_EARLY}: Payload MUST be padded to PAYLOAD\_LEN with padding
bytes. Padding bytes SHOULD be set to random values.\\\\
Other \textbf{fixed-length} cells:\\\\
Payload MUST be padded to PAYLOAD\_LEN with padding bytes.
Padding bytes SHOULD be set to NULL.\\
We recommend random padding in \textit{RELAY/RELAY\_EARLY} cells, so that the cell
content is unpredictable. \textbf{See proposal 289 for details} (in short, the traffic can not be easily identified).\\ For other
cells, TLS authenticates cell content, so randomized padding bytes are
redundant.\\
\\
\\
\textbf{Other commands explanation}
\\
\\
\textit{PADDING} cells are currently used to implement connection keepalive.
If there is no other traffic, ORs and OPs send one another a PADDING
cell every few minutes.\\
\textit{CREATE, CREATE2, CREATED, CREATED2,} and DESTROY cells are used to
manage circuits; (see section 5 below)
\textit{RELAY} cells are used to send commands and data along a circuit; (see
section 6 below)\\
\textit{VERSIONS and NETINFO} cells are used to set up connections in link
protocols v2 and higher; in link protocol v3 and higher, \textit{CERTS,
AUTH\_CHALLENGE}, and \textit{AUTHENTICATE} may also be used. (See section 4
below)\\
\\
