\section{Application connections and stream management}

\subsection{Relay cells}

Within a circuit, the OP and the end node use the contents of
RELAY packets to tunnel end-to-end commands and TCP connections
("Streams") across circuits. End-to-end commands can be initiated
by either edge; streams are initiated by the OP.

\paragraph{}
End nodes that accept streams may be:
\begin{itemize}
    \item exit relays (RELAY_BEGIN, anonymous),
    \item directory servers (RELAY_BEGIN_DIR, anonymous or non-anonymous),
    \item onion services (RELAY_BEGIN, anonymous via a rendezvous point).
\end{itemize}

\paragraph{}
The payload of each unencrypted RELAY cell consists of:

\begin{verbatim}
    Relay command [1 byte]
    'Recognized' [2 bytes]
    StreamID [2 bytes]
    Digest [4 bytes]
    Length [2 bytes]
    Data [PAYLOAD_LEN-11 bytes]
\end{verbatim}

\paragraph{}
The relay commands are:

\begin{verbatim}
    1 -- RELAY_BEGIN [forward]
    2 -- RELAY_DATA [forward or backward]
    3 -- RELAY_END [forward or backward]
    4 -- RELAY_CONNECTED [backward]
    5 -- RELAY_SENDME [forward or backward] [sometimes control]
    6 -- RELAY_EXTEND [forward] [control]
    7 -- RELAY_EXTENDED [backward] [control]
    8 -- RELAY_TRUNCATE [forward] [control]
    9 -- RELAY_TRUNCATED [backward] [control]
    10 -- RELAY_DROP [forward or backward] [control]
    11 -- RELAY_RESOLVE [forward]
    12 -- RELAY_RESOLVED [backward]
    13 -- RELAY_BEGIN_DIR [forward]
    14 -- RELAY_EXTEND2 [forward] [control]
    15 -- RELAY_EXTENDED2 [backward] [control]
    32..40 -- Used for hidden services; see rend-spec-{v2,v3}.txt.
\end{verbatim}

\paragraph{}
Commands labelled as "forward" must only be sent by the originator
of the circuit. Commands labelled as "backward" must only be sent by
other nodes in the circuit back to the originator. Commands marked
as either can be sent either by the originator or other nodes.
The 'recognized' field is used as a simple indication that the cell
is still encrypted. It is an optimization to avoid calculating
expensive digests for every cell. When sending cells, the unencrypted
'recognized' MUST be set to zero.

\paragraph{}
When receiving and decrypting cells the 'recognized' will always be
zero if we're the endpoint that the cell is destined for. For cells
that we should relay, the 'recognized' field will usually be nonzero,
but will accidentally be zero with $P=2^{-16}$.

\paragraph{}
When handling a relay cell, if the 'recognized' in field in a
decrypted relay payload is zero, the 'digest' field is computed as
the first four bytes of the running digest of all the bytes that have
been destined for this hop of the circuit or originated from this hop
of the circuit, seeded from Df or Db respectively, and including this
RELAY cell's entire payload (taken with the digest field set to zero).
If the digest is correct, the cell is considered "recognized" for the
purposes of decryption.

\paragraph{}
(The digest does not include any bytes from relay cells that do
not start or end at this hop of the circuit. That is, it does not
include forwarded data. Therefore if 'recognized' is zero but the
digest does not match, the running digest at that node should
not be updated, and the cell should be forwarded on.)

\paragraph{}
All RELAY cells pertaining to the same tunneled stream have the same
stream ID. StreamIDs are chosen arbitrarily by the OP. No stream
may have a StreamID of zero. Rather, RELAY cells that affect the
entire circuit rather than a particular stream use a StreamID of zero
-- they are marked in the table above as "[control]" style
cells. (Sendme cells are marked as "sometimes control" because they
can include a StreamID or not depending on their purpose.)

\paragraph{}
The 'Length' field of a relay cell contains the number of bytes in
the relay payload which contain real payload data. The remainder of
the unencrypted payload is padded with padding bytes. Implementations
handle padding bytes of unencrypted relay cells as they do padding
bytes for other cell types;

\paragraph{}
If the RELAY cell is recognized but the relay command is not
understood, the cell must be dropped and ignored. Its contents
still count with respect to the digests and flow control windows, though.

\subsection{Opening streams and transferring data}
To open a new anonymized TCP connection, the OP chooses an open
circuit to an exit that may be able to connect to the destination
address, selects an arbitrary StreamID not yet used on that circuit,
and constructs a RELAY_BEGIN cell with a payload encoding the address
and port of the destination host. The payload format is:
\begin{verbatim}
    ADDRPORT [nul-terminated string]
    FLAGS [4 bytes]
    ADDRPORT is made of ADDRESS | ':' | PORT | [00]
\end{verbatim}

\paragraph{}
where ADDRESS can be a DNS hostname, or an IPv4 address in
dotted-quad format, or an IPv6 address surrounded by square brackets;
and where PORT is a decimal integer between 1 and 65535, inclusive.
The FLAGS value has one or more of the following bits set, where
"bit 01" is the LSB of the 32-bit value, and "bit 32" is the MSB.
(Remember that all values in Tor are big-endian, so
the MSB of a 4-byte value is the MSB of the first byte, and the LSB
of a 4-byte value is the LSB of its last byte.)
bit meaning
1 -- IPv6 okay. We support learning about IPv6 addresses and
connecting to IPv6 addresses.
2 -- IPv4 not okay. We don't want to learn about IPv4 addresses
or connect to them.
3 -- IPv6 preferred. If there are both IPv4 and IPv6 addresses,
we want to connect to the IPv6 one. (By default, we connect
to the IPv4 address.)
4..32 -- Reserved. Current clients MUST NOT set these. Servers
MUST ignore them.
Upon receiving this cell, the exit node resolves the address as
necessary, and opens a new TCP connection to the target port. If the
address cannot be resolved, or a connection can't be established, the
exit node replies with a RELAY_END cell. (See 6.4 below.)
Otherwise, the exit node replies with a RELAY_CONNECTED cell, whose
payload is in one of the following formats:
The IPv4 address to which the connection was made [4 octets]
A number of seconds (TTL) for which the address may be cached [4 octets]
or
Four zero-valued octets [4 octets]
An address type (6) [1 octet]
The IPv6 address to which the connection was made [16 octets]
A number of seconds (TTL) for which the address may be cached [4 octets]
[Tor exit nodes before 0.1.2.0 set the TTL field to a fixed value. Later
versions set the TTL to the last value seen from a DNS server, and expire
their own cached entries after a fixed interval. This prevents certain
attacks.]
Once a connection has been established, the OP and exit node
package stream data in RELAY_DATA cells, and upon receiving such
cells, echo their contents to the corresponding TCP stream.
If the exit node does not support optimistic data (i.e. its
version number is before 0.2.3.1-alpha), then the OP MUST wait
for a RELAY_CONNECTED cell before sending any data. If the exit
node supports optimistic data (i.e. its version number is
0.2.3.1-alpha or later), then the OP MAY send RELAY_DATA cells
immediately after sending the RELAY_BEGIN cell (and before
receiving either a RELAY_CONNECTED or RELAY_END cell).
RELAY_DATA cells sent to unrecognized streams are dropped. If
the exit node supports optimistic data, then RELAY_DATA cells it
receives on streams which have seen RELAY_BEGIN but have not yet
been replied to with a RELAY_CONNECTED or RELAY_END are queued.
If the stream creation succeeds with a RELAY_CONNECTED, the queue
is processed immediately afterwards; if the stream creation fails
with a RELAY_END, the contents of the queue are deleted.
Relay RELAY_DROP cells are long-range dummies; upon receiving such
a cell, the OR or OP must drop it.
6.2.1. Opening a directory stream
If a Tor relay is a directory server, it should respond to a
RELAY_BEGIN_DIR cell as if it had received a BEGIN cell requesting a
connection to its directory port. RELAY_BEGIN_DIR cells ignore exit
policy, since the stream is local to the Tor process.
Directory servers may be:
* authoritative directories (RELAY_BEGIN_DIR, usually non-anonymous),
* bridge authoritative directories (RELAY_BEGIN_DIR, anonymous),
* directory mirrors (RELAY_BEGIN_DIR, usually non-anonymous),
* onion service directories (RELAY_BEGIN_DIR, anonymous).
If the Tor relay is not running a directory service, it should respond
with a REASON_NOTDIRECTORY RELAY_END cell.
Clients MUST generate an all-zero payload for RELAY_BEGIN_DIR cells,
and relays MUST ignore the payload.
In response to a RELAY_BEGIN_DIR cell, relays respond either with a
RELAY_CONNECTED cell on succcess, or a RELAY_END cell on failure. They
MUST send a RELAY_CONNECTED cell all-zero payload, and clients MUST ignore
the payload.
[RELAY_BEGIN_DIR was not supported before Tor 0.1.2.2-alpha; clients
SHOULD NOT send it to routers running earlier versions of Tor.]
6.3. Closing streams
When an anonymized TCP connection is closed, or an edge node
encounters error on any stream, it sends a 'RELAY_END' cell along the
circuit (if possible) and closes the TCP connection immediately. If
an edge node receives a 'RELAY_END' cell for any stream, it closes
the TCP connection completely, and sends nothing more along the
circuit for that stream.
The payload of a RELAY_END cell begins with a single 'reason' byte to
describe why the stream is closing. For some reasons, it contains
additional data (depending on the reason.) The values are:
1 -- REASON_MISC (catch-all for unlisted reasons)
2 -- REASON_RESOLVEFAILED (couldn't look up hostname)
3 -- REASON_CONNECTREFUSED (remote host refused connection) [*]
4 -- REASON_EXITPOLICY (OR refuses to connect to host or port)
5 -- REASON_DESTROY (Circuit is being destroyed)
6 -- REASON_DONE (Anonymized TCP connection was closed)
7 -- REASON_TIMEOUT (Connection timed out, or OR timed out
while connecting)
8 -- REASON_NOROUTE (Routing error while attempting to
contact destination)
9 -- REASON_HIBERNATING (OR is temporarily hibernating)
10 -- REASON_INTERNAL (Internal error at the OR)
11 -- REASON_RESOURCELIMIT (OR has no resources to fulfill request)
12 -- REASON_CONNRESET (Connection was unexpectedly reset)
13 -- REASON_TORPROTOCOL (Sent when closing connection because of
Tor protocol violations.)
14 -- REASON_NOTDIRECTORY (Client sent RELAY_BEGIN_DIR to a
non-directory relay.)
(With REASON_EXITPOLICY, the 4-byte IPv4 address or 16-byte IPv6 address
forms the optional data, along with a 4-byte TTL; no other reason
currently has extra data.)
OPs and ORs MUST accept reasons not on the above list, since future
versions of Tor may provide more fine-grained reasons.
Tors SHOULD NOT send any reason except REASON_MISC for a stream that they
have originated.
[*] Older versions of Tor also send this reason when connections are
reset.
--- [The rest of this section describes unimplemented functionality.]
Because TCP connections can be half-open, we follow an equivalent
to TCP's FIN/FIN-ACK/ACK protocol to close streams.
An exit (or onion service) connection can have a TCP stream in one of
three states: 'OPEN', 'DONE_PACKAGING', and 'DONE_DELIVERING'. For the
purposes of modeling transitions, we treat 'CLOSED' as a fourth state,
although connections in this state are not, in fact, tracked by the
onion router.
A stream begins in the 'OPEN' state. Upon receiving a 'FIN' from
the corresponding TCP connection, the edge node sends a 'RELAY_FIN'
cell along the circuit and changes its state to 'DONE_PACKAGING'.
Upon receiving a 'RELAY_FIN' cell, an edge node sends a 'FIN' to
the corresponding TCP connection (e.g., by calling
shutdown(SHUT_WR)) and changing its state to 'DONE_DELIVERING'.
When a stream in already in 'DONE_DELIVERING' receives a 'FIN', it
also sends a 'RELAY_FIN' along the circuit, and changes its state
to 'CLOSED'. When a stream already in 'DONE_PACKAGING' receives a
'RELAY_FIN' cell, it sends a 'FIN' and changes its state to
'CLOSED'.
If an edge node encounters an error on any stream, it sends a
'RELAY_END' cell (if possible) and closes the stream immediately.
6.4. Remote hostname lookup
To find the address associated with a hostname, the OP sends a
RELAY_RESOLVE cell containing the hostname to be resolved with a NUL
terminating byte. (For a reverse lookup, the OP sends a RELAY_RESOLVE
cell containing an in-addr.arpa address.) The OR replies with a
RELAY_RESOLVED cell containing any number of answers. Each answer is
of the form:
Type (1 octet)
Length (1 octet)
Value (variable-width)
TTL (4 octets)
"Length" is the length of the Value field.
"Type" is one of:
0x00 -- Hostname
0x04 -- IPv4 address
0x06 -- IPv6 address
0xF0 -- Error, transient
0xF1 -- Error, nontransient
If any answer has a type of 'Error', then no other answer may be
given.
The 'Value' field encodes the answer:
IP addresses are given in network order.
Hostnames are given in standard DNS order ("www.example.com")
and not NUL-terminated.
The content of Errors is currently ignored. Relays currently
set it to the string "Error resolving hostname" with no
terminating NUL. Implementations MUST ignore this value.
For backward compatibility, if there are any IPv4 answers, one of those
must be given as the first answer.
The RELAY_RESOLVE cell must use a nonzero, distinct streamID; the
corresponding RELAY_RESOLVED cell must use the same streamID. No stream
is actually created by the OR when resolving the name.