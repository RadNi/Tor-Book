\chapter{Negotiating and initializing connections}


\paragraph{}
After Tor instances negotiate handshake with either the
"renegotiation"(v2) or "in-protocol"(v3) handshakes, they must exchange
a number of cells to set up the Tor connection  and make it
open and usable for circuits.

\paragraph{}
When the \textbf{renegotiation} handshake is used,
\begin{itemize}
    \item Both parties immediately send a VERSIONS cell.
    \item A link protocol version is negotiated (which will be 2).
    \item Each send a NETINFO cell to confirm their addresses and timestamps.
\end{itemize}
No other intervening cell types are allowed.

\paragraph{}
When the \textbf{in-protocol} handshake is used,
\begin{itemize}
    \item the initiator sends a VERSIONS cell to indicate that it will not be renegotiating.
    \item The responder sends
    \begin{itemize}
        \item a VERSIONS cell, a CERTS cell to give the
        initiator the certificates it needs to learn the responder's
        identity.
        \item an AUTH\_CHALLENGE cell that the initiator must include
        as part of its answer if it chooses to authenticate.
        \item a NETINFO cell
    \end{itemize}
    \item As soon as it gets the CERTS cell, the initiator knows
    whether the responder is correctly authenticated. At this point the
    initiator behaves differently depending on whether it wants to
    authenticate or not.
    \begin{itemize}
        \item If it does not want to authenticate, it MUST send a NETINFO cell.
        \item If it does want to authenticate, it MUST send a CERTS cell,
        an AUTHENTICATE cell, and a NETINFO cell.
    \end{itemize}
\end{itemize}
When this handshake is in use, the first cell must be VERSIONS, VPADDING, or
AUTHORIZE, and no other cell type is allowed to intervene besides
those specified, except for VPADDING cells.

\paragraph{}
[Tor versions before 0.2.3.11-alpha did not recognize the AUTHORIZE cell,
and did not permit any command other than VERSIONS as the first cell of
the in-protocol handshake.]


\section{Negotiating versions with VERSIONS cells}
There are multiple instances of the Tor link connection protocol.
\begin{itemize}
    \item
    Any connection negotiated using the "certificates up front" handshake (see
    section 2 above) is "version 1".
    \item In any connection where both parties
    have behaved as in the "renegotiation" handshake, the link protocol
    version must be 2.
    \item In any connection where both parties have behaved
    as in the "in-protocol" handshake, the link protocol must be 3 or higher.
\end{itemize}

\paragraph{}
To determine the version, in any connection where the "renegotiation"
or "in-protocol" handshake was used (that is, where the responder
sent only one certificate at first and where the initiator did not
send any certificates in the first negotiation), both parties MUST
send a VERSIONS cell.
\begin{itemize}
    \item
    In "renegotiation", they send a VERSIONS cell
    right after the renegotiation is finished, before any other cells are
    sent.
    \item In "in-protocol", the initiator sends a VERSIONS cell
    immediately after the initial TLS handshake, and the responder
    replies immediately with a VERSIONS cell. (As an exception to this rule,
    if both sides support the "in-protocol" handshake, either side may send
    VPADDING cells at any time.)
\end{itemize}

\paragraph{}
The payload in a VERSIONS cell is a series of big-endian two-byte
integers. Both parties MUST select as the link protocol version the
highest number contained both in the VERSIONS cell they sent and in the
versions cell they received. If they have no such version in common,
they cannot communicate and MUST close the connection. Either party MUST
close the connection if the versions cell is not well-formed (for example,
if it contains an odd number of bytes)

\paragraph{}
Any VERSIONS cells sent after the first VERSIONS cell MUST be ignored.
(To be interpreted correctly, later VERSIONS cells MUST have a CIRCID\_LEN
matching the version negotiated with the first VERSIONS cell.)


\paragraph{}
Since the version 1 link protocol does not use the "renegotiation"
handshake, implementations MUST NOT list version 1 in their VERSIONS
cell. When the "renegotiation" handshake is used, implementations
MUST list only the version 2. When the "in-protocol" handshake is
used, implementations MUST NOT list any version before 3, and SHOULD
list at least version 3.

\paragraph{}
Link protocols differences are:
\begin{enumerate}
    \item The "certs up front" handshake.
    \item Uses the renegotiation-based handshake. Introduces
          variable-length cells.
    \item Uses the in-protocol handshake.
    \item Increases circuit ID width to 4 bytes.
    \item Adds support for link padding and negotiation (padding-spec.txt).
\end{enumerate}


\section{CERTS cells}
The CERTS cell describes the keys that a Tor instance is claiming
to have. It is a variable-length cell. Its payload format is:

\begin{verbatim}
N: Number of certs in cell [1 octet]
N times:    CertType [1 octet]
            CLEN [2 octets]
            Certificate [CLEN octets]
\end{verbatim}

\paragraph{}
Any extra octets at the end of a CERTS cell MUST be ignored.

\paragraph{}
Relevant certType values are:
\begin{enumerate}
    \item Link key certificate certified by RSA1024 identity
    \item RSA1024 Identity certificate, self-signed.
    \item RSA1024 AUTHENTICATE cell link certificate, signed with RSA1024 key.
    \item Ed25519 signing key, signed with identity key.
    \item TLS link certificate, signed with ed25519 signing key.
    \item Ed25519 AUTHENTICATE cell key, signed with ed25519 signing key.
    \item Ed25519 identity, signed with RSA identity.
\end{enumerate}

\paragraph{}
The certificate format for certificate types 1-3 is DER encoded
X509. For types 4-6 ED25519 encoding is used. For type 7 RSA CrossCert
encoding is used.

\paragraph{X509 certificated encoding}

In cryptography, X.509 is a standard defining the format of public key certificates.
X.509 certificates are used in many Internet protocols, including TLS/SSL, which is
the basis for HTTPS, the secure protocol for browsing the web. They are also used
in offline applications, like electronic signatures. An X.509 certificate contains a
public key and an identity (a hostname, or an organization, or an individual), and
is either signed by a certificate authority or self-signed. When a certificate is
signed by a trusted certificate authority, or validated by other means, someone holding
that certificate can rely on the public key it contains to establish secure communications
with another party, or validate documents digitally signed by the corresponding private key.

\paragraph{ED25519 encoding}

This encoding has the following structure:

\begin{verbatim}
struct ed25519_cert_st {
    uint8_t version;
    uint8_t cert_type;
    uint32_t exp_field;
    uint8_t cert_key_type;
    uint8_t certified_key[32];
    uint8_t n_extensions;
    TRUNNEL_DYNARRAY_HEAD(, struct ed25519_cert_extension_st *) ext;
    uint8_t signature[64];
    uint8_t trunnel_error_code_;
};
\end{verbatim}

\paragraph{}
Note: A CERTS cell may have no more than one certificate of each CertType.

\paragraph{}
To authenticate \textbf{the responder}, \textbf{the initiator} must check a set of conditions ,
based on which of these cases has happened:

\begin{enumerate}
    \item having a given Ed25519, RSA identity key combination
        \begin{itemize}
            \item The CERTS cell contains exactly one CertType 2 "ID" certificate.
            \item The CERTS cell contains exactly one CertType 4 Ed25519
            "Id->Signing" cert.
            \item The CERTS cell contains exactly one CertType 5 Ed25519
            "Signing->link" certificate.
            \item The CERTS cell contains exactly one CertType 7 "RSA->Ed25519"
            cross-certificate.
            \item All X.509 certificates above have validAfter and validUntil dates;
            no X.509 or Ed25519 certificates are expired.
            \item All certificates are correctly signed.
            \item The certified key in the Signing->Link certificate matches the
            SHA256 digest of the certificate that was used to
            authenticate the TLS connection.
            \item The identity key listed in the ID->Signing cert was used to
            sign the ID->Signing Cert.
            \item The Signing->Link cert was signed with the Signing key listed
            in the ID->Signing cert.
            \item The RSA->Ed25519 cross-certificate certifies the Ed25519
            identity, and is signed with the RSA identity listed in the
            "ID" certificate.
            \item The certified key in the ID certificate is a 1024-bit RSA key.
            \item The RSA ID certificate is correctly self-signed. 
        \end{itemize}
    \item having a given RSA identity only
        \begin{itemize}
            \item The CERTS cell contains exactly one CertType 1 "Link" certificate.
            \item The CERTS cell contains exactly one CertType 2 "ID" certificate.
            \item Both certificates have validAfter and validUntil dates that
            are not expired.
            \item The certified key in the Link certificate matches the
            link key that was used to negotiate the TLS connection.
            \item The certified key in the ID certificate is a 1024-bit RSA key.
            \item The certified key in the ID certificate was used to sign both
            certificates.
            \item The link certificate is correctly signed with the key in the
            ID certificate
            \item The ID certificate is correctly self-signed
        \end{itemize}
\end{enumerate}


\paragraph{}
In both cases above, checking these conditions is sufficient to
authenticate that the initiator is talking to the Tor node with the
expected identity, as certified in the ID certificate(s).

\paragraph{}
To authenticate \textbf{the initiator}, \textbf{the responder} must check a set of conditions ,
based on which of these cases has happened:

\begin{enumerate}
    \item having a given Ed25519, RSA identity key combination
        \begin{itemize}
            \item The CERTS cell contains exactly one CertType 2 "ID" certificate.
            \item The CERTS cell contains exactly one CertType 4 Ed25519
            "Id->Signing" certificate.
            \item The CERTS cell contains exactly one CertType 6 Ed25519
            "Signing->auth" certificate.
            \item The CERTS cell contains exactly one CertType 7 "RSA->Ed25519"
            cross-certificate.
            \item All X.509 certificates above have validAfter and validUntil dates;
            no X.509 or Ed25519 certificates are expired.
            \item All certificates are correctly signed.
            \item The identity key listed in the ID->Signing cert was used to
            sign the ID->Signing Cert.
            \item The Signing->AUTH cert was signed with the Signing key listed
            in the ID->Signing cert.
            \item The RSA->Ed25519 cross-certificate certifies the Ed25519
            identity, and is signed with the RSA identity listed in the
            "ID" certificate.
            \item The certified key in the ID certificate is a 1024-bit RSA key.
            \item The RSA ID certificate is correctly self-signed.
        \end{itemize}
    \item having a given RSA identity only
        \begin{itemize}
            \item The CERTS cell contains exactly one CertType 3 "AUTH" certificate.
            \item The CERTS cell contains exactly one CertType 2 "ID" certificate.
            \item Both certificates have validAfter and validUntil dates that
            are not expired.
            \item The certified key in the AUTH certificate is a 1024-bit RSA key.
            \item The certified key in the ID certificate is a 1024-bit RSA key.
            \item The certified key in the ID certificate was used to sign both
            certificates.
            \item The auth certificate is correctly signed with the key in the
            ID certificate.
            \item The ID certificate is correctly self-signed.
        \end{itemize}
\end{enumerate}

\paragraph{}
Checking these conditions is NOT sufficient to authenticate that the
initiator has the ID it claims; to do so, AUTH\_CHALLENGE and AUTHENTICATE cells
must be exchanged.

\section{AUTH\_CHALLENGE cells}

An AUTH\_CHALLENGE cell is a variable-length cell with the following
fields:

\begin{verbatim}
    Challenge [32 octets]
    N_Methods [2 octets]
    Methods   [2 * N_Methods octets]
\end{verbatim}

\paragraph{}
It is sent from the responder to the initiator. Initiators MUST
ignore unexpected bytes at the end of the cell. Responders MUST
generate every challenge independently using a strong RNG or PRNG.

\paragraph{}
The Challenge field is a randomly generated string that the
initiator must sign (a hash of) as part of authenticating. The
methods are the authentication methods that the responder will
accept. Only two authentication methods are defined right now:
RSA-SHA256-TLSSecret and Ed25519-SHA256-RFC5705.

\section{AUTHENTICATE cells}

If an initiator wants to authenticate, it responds to the
AUTH\_CHALLENGE cell with a CERTS cell and an AUTHENTICATE cell.
The CERTS cell is as a server would send, except that instead of
sending a CertType 1 (and possibly CertType 5) certs for arbitrary link
certificates, the initiator sends a CertType 3 (and possibly
CertType 6) cert for an RSA/Ed25519 AUTHENTICATE key.
This difference is because we allow any link key type on a TLS
link, but the protocol described here will only work for specific key
types as described in 4.4.1 and 4.4.2 below.

\paragraph{}
An AUTHENTICATE cell contains the following:
\begin{verbatim}
    AuthType [2 octets]
    AuthLen [2 octets]
    Authentication [AuthLen octets]
\end{verbatim}

\paragraph{}
Responders MUST ignore extra bytes at the end of an AUTHENTICATE
cell. Recognized AuthTypes are 1 and 3, described in the next
two sections.

\paragraph{}
Initiators MUST NOT send an AUTHENTICATE cell before they have
verified the certificates presented in the responder's CERTS
cell, and authenticated the responder.

\subsection{Link authentication type 1: RSA-SHA256-TLSSecret}

If AuthType is 1 (meaning "RSA-SHA256-TLSSecret"), then the
Authentication field of the AUTHENTICATE cell contains the following:

\begin{itemize}
    \item TYPE: The characters "AUTH0001" [8 octets]
    \item CID: A SHA256 hash of the initiator's RSA1024 identity key [32 octets]
    \item SID: A SHA256 hash of the responder's RSA1024 identity key [32 octets]
    \item SLOG: A SHA256 hash of all bytes sent from the responder to the
    initiator as part of the negotiation up to and including the
    AUTH\_CHALLENGE cell; that is, the VERSIONS cell, the CERTS cell,
    the AUTH\_CHALLENGE cell, and any padding cells. [32 octets]
    \item CLOG: A SHA256 hash of all bytes sent from the initiator to the
    responder as part of the negotiation so far; that is, the
    VERSIONS cell and the CERTS cell and any padding cells. [32
    octets]
    \item SCERT: A SHA256 hash of the responder's TLS link certificate. [32
    octets]
    \item TLSSECRETS: A SHA256 HMAC, using the TLS master secret as the
    secret key, of the following:
    \begin{itemize}
        \item client\_random, as sent in the TLS Client Hello
        \item server\_random, as sent in the TLS Server Hello
        \item the NUL terminated ASCII string:
        \newline
        "Tor V3 handshake TLS cross-certification"
    \end{itemize} [32 octets]
    \item RAND: A 24 byte value, randomly chosen by the initiator. (In an
    imitation of SSL3's gmt\_unix\_time field, older versions of Tor
    sent an 8-byte timestamp as the first 8 bytes of this field;
    new implementations should not do that.) [24 octets]
    \item SIG: A signature of a SHA256 hash of all the previous fields
    using the initiator's "Authenticate" key as presented. (As
    always in Tor, we use OAEP-MGF1 padding; see tor-spec.txt
    section 0.3.)
    [variable length]
\end{itemize}

\paragraph{}
To check the AUTHENTICATE cell, a responder checks that all fields
from TYPE through TLSSECRETS contain their unique
correct values as described above, and then verifies the signature.
The server MUST ignore any extra bytes in the signed data after
the RAND field.

\paragraph{}
Responders MUST NOT accept this AuthType if the initiator has
claimed to have an Ed25519 identity.

\subsection{Link authentication type 3: Ed25519-SHA256-RFC5705}

\paragraph{}
If AuthType is 3, meaning "Ed25519-SHA256-RFC5705", the
Authentication field of the AuthType cell is as below:

\paragraph{}
Modified values and new fields below are marked with asterisks.

\begin{itemize}
    \item TYPE: The characters "AUTH0003" [8 octets]
    \item CID: A SHA256 hash of the initiator's RSA1024 identity key [32 octets]
    \item SID: A SHA256 hash of the responder's RSA1024 identity key [32 octets]
    \item CID\_ED: The initiator's Ed25519 identity key [32 octets]
    \item SID\_ED: The responder's Ed25519 identity key, or all-zero. [32 octets]
    \item SLOG: A SHA256 hash of all bytes sent from the responder to the
    initiator as part of the negotiation up to and including the
    AUTH\_CHALLENGE cell; that is, the VERSIONS cell, the CERTS cell,
    the AUTH\_CHALLENGE cell, and any padding cells. [32 octets]
    \item CLOG: A SHA256 hash of all bytes sent from the initiator to the
    responder as part of the negotiation so far; that is, the
    VERSIONS cell and the CERTS cell and any padding cells. [32
    octets]
    \item SCERT: A SHA256 hash of the responder's TLS link certificate. [32
    octets]
    \item TLSSECRETS: The output of an RFC5705 Exporter function on the
    TLS session, using as its inputs:
    \begin{itemize}
        \item The label string "EXPORTER FOR TOR TLS CLIENT BINDING AUTH0003"
        \item The context value equal to the initiator's Ed25519 identity key.
        \item The length 32.
    \end{itemize}[32 octets]
    \item RAND: A 24 byte value, randomly chosen by the initiator. [24 octets]
    \item SIG: A signature of all previous fields using the initiator's
    Ed25519 authentication key (as in the cert with CertType 6).
    [variable length]
\end{itemize}

\paragraph{}
To check the AUTHENTICATE cell, a responder checks that all fields
from TYPE through TLSSECRETS contain their unique
correct values as described above, and then verifies the signature.
The server MUST ignore any extra bytes in the signed data after
the RAND field.

\section{NETINFO cells}

If version 2 or higher is negotiated, each party sends the other a
NETINFO cell. The cell's payload is:

\begin{verbatim}
    TIME (Timestamp) [4 bytes]
    OTHERADDR (Other OR's address) [variable]
        ATYPE (Address type) [1 byte]
        ALEN (Adress length) [1 byte]
        AVAL (Address value in NBO) [ALEN bytes]
    NMYADDR (Number of this OR's addresses) [1 byte]
    NMYADDR times:
        ATYPE (Address type) [1 byte]
        ALEN (Adress length) [1 byte]
        AVAL (Address value in NBO)) [ALEN bytes]
\end{verbatim}

\paragraph{}
Recognized address types (ATYPE) are:
\begin{verbatim}
    [04] IPv4.
    [06] IPv6.
\end{verbatim}

\paragraph{}
ALEN MUST be 4 when ATYPE is 0x04 (IPv4) and 16 when ATYPE is 0x06
(IPv6).

\paragraph{}
The timestamp is a big-endian unsigned integer number of seconds
since the Unix epoch. Implementations MUST ignore unexpected bytes
at the end of the cell.

\paragraph{}
Implementations MAY use the timestamp value to help decide if their
clocks are skewed. Initiators MAY use "other OR's address" to help
learn which address their connections may be originating from, if they do
not know it; and to learn whether the peer will treat the current
connection as canonical. Implementations SHOULD NOT trust these
values unconditionally, especially when they come from non-authorities,
since the other party can lie about the time or IP addresses it sees.

\paragraph{}
Initiators SHOULD use "this OR's address" to make sure
that they have connected to another OR at its canonical address.
